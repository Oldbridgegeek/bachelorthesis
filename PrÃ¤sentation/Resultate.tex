\begin{frame}
\frametitle{Zusammenfassung}
Wir haben zwei Möglichkeiten betrachtet die Pseudoinverse der zu untersuchunden Bilinearformen herzuleiten:
\begin{enumerate}
\item Ausnutzung der Tensorstruktur und Herleitung einer einfachen Form
\item Universellen Ansatz über die HOSVD
\end{enumerate}
Die beiden Methoden haben ihre Vor und Nachteile. Die erste Methode:
\begin{itemize}
\item nicht flexibel
\item individueller Lösungsansatz für jede Bilinearform
\item Komplexitätsklasse $O(N^3)$
\end{itemize}
Die zweite Methode über die HOSVD:
\begin{itemize}
\item flexibilität
\item Komplexitätsklasse vom $O(N^5)$
\end{itemize}
\end{frame}
\begin{frame}
\frametitle{Zukunft}
In Zukunft kann man:
\begin{itemize}
\item Mehr Bilinearformen auf Tensorstrukturen untersuchen
\item Das Kommunikationsverhalten des $n-mode$ Produkt  mit anderen Operatoren näher betrachten
\item Effizientere Matrix-Matrix Multiplikation
\item Unsere definierte Entfaltung bezüglich der HOSVD untersuchen
\begin{equation*}
\mathcal{X}^{(p)} v = [(U^{(1)} \otimes U^{(2)}) \mathcal{G}^{(p)} (U^{(3)} \otimes U^{(4)})] v
\end{equation*}
\end{itemize}
\end{frame}