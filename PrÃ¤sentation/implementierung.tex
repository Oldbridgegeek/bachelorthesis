\subsection{Effiziente Algorithmen}
\begin{frame}
\frametitle{Kronecker Produkt}

Wir wollen folgende zwei Aufgaben effizient lösen:
\begin{enumerate}
\item $ z = (\mathcal{B} \otimes \mathcal{A}) y$
\item $ z = (\mathcal{C} \otimes \mathcal{B} \otimes \mathcal{A}) v $
\end{enumerate}

\end{frame}

\begin{frame}
\frametitle{ $ z = (\mathcal{B} \otimes \mathcal{A}) y$ }
\begin{equation*}
\mathcal{B} \times \mathcal{A} =
\begin{pmatrix}
b_{11}\mathcal{A} & \dots  & b_{1q}\mathcal{A} \\
\vdots & \ddots & \vdots \\
b_{p1}\mathcal{A} & \dots & b_{pq}\mathcal{A} \\
\end{pmatrix}.
\end{equation*}

\begin{enumerate}
\item Geordnete Indexierung $y = (y_1,y_2,\dots,y_n,\dots,\dots,y_{(q-1)n+1},y_{(q-1)n+2},\dots,y_{qn})^T$ 
\item Definiere $y^{(i)}=(y_1,y_2,\dots,y_{in})^T$
\item Berechne $w^{(i)} = \mathcal{A}y^{(i)}$
\item Berechne $ z^{(k)}_j = \sum_{i=1}^{q} b_{ki} w_j^{(i)}$
\end{enumerate}

\end{frame}


\begin{frame}
\begin{mdframed}[backgroundcolor=blue!3] 
\begin{algorithmic}
\For {i=1 < N }
	\For {j= 1 < N }
			\State $w^{(i)}_{j} = y^{(i)}_1 a_{j1} + \dots + y^{(i)}_n a_{jn}$
	\EndFor
\EndFor
\For {k=1 < N }
	\For {j= 1 < N }
			\State $z^{(k)}_j := w^{(1)}_j b_{k1} + \dots + w^{(q)}_j b_{kq}$
	\EndFor
\EndFor

\end{algorithmic}
\end{mdframed}




\end{frame}


\begin{frame}
\frametitle{ $ z = (\mathcal{C} \otimes \mathcal{B} \otimes \mathcal{A}) v $ }
\begin{equation*} 
\begin{pmatrix}
c_{11} \textcolor{green}{b}_{11} \textcolor{red}{\mathcal{A}} & \dots  & c_{11} \textcolor{green}{b}_{1N} \textcolor{red}{\mathcal{A}} & \dots & \dots & c_{1N}\textcolor{green}{b}_{11}\textcolor{red}{\mathcal{A}} & \dots & c_{1N}\textcolor{green}{b}_{1N}\textcolor{red}{\mathcal{A}}  \\

\vdots & \vdots & \ddots & \ddots  & \ddots & \vdots & \vdots & \vdots \\
c_{11} \textcolor{green}{b}_{N 1} \textcolor{red}{\mathcal{A}} & \dots  & c_{11} \textcolor{green}{b}_{N N} \textcolor{red}{\mathcal{A}} & \dots & \dots & c_{1 N}\textcolor{green}{b}_{N 1}\textcolor{red}{\mathcal{A}} & \dots & c_{1 N}\textcolor{green}{b}_{N N}\textcolor{red}{\mathcal{A}}  \\
c_{21} \textcolor{green}{b}_{11} \textcolor{red}{\mathcal{A}} & \dots  & c_{21} \textcolor{green}{b}_{1N} \textcolor{red}{\mathcal{A}} & \dots & \dots & c_{2N}\textcolor{green}{b}_{11}\textcolor{red}{\mathcal{A}} & \dots & c_{2 N}\textcolor{green}{b}_{1 N}\textcolor{red}{\mathcal{A}}  \\
\vdots & \vdots & \ddots & \ddots  & \ddots & \vdots & \vdots & \vdots \\
c_{N 1} \textcolor{green}{b}_{N 1} \textcolor{red}{\mathcal{A}} & \dots  & c_{N 1} \textcolor{green}{b}_{N N} \textcolor{red}{\mathcal{A}} & \dots & \dots & c_{N N}\textcolor{green}{b}_{N 1}\textcolor{red}{\mathcal{A}} & \dots & c_{N N}\textcolor{green}{b}_{N N}\textcolor{red}{\mathcal{A}}  \\
\end{pmatrix} 
\end{equation*}
\end{frame}

\begin{frame}

\begin{mdframed}[backgroundcolor=blue!3] 
\begin{algorithmic}
\For {k=1 < N }
	\For {i= 1 < N }
		\For {j= 1 < N }
			\State $w_{k}(i,j) = \mathcal{V}(i,j,1)a_{k1} + \dots + \mathcal{V}(i,j,N)a_{kN}$
		\EndFor
	\EndFor
\EndFor
\For {k=1 < N }
	\For {i= 1 < N }
		\For {j= 1 < N }
			\State $\mathcal{W}_{i,j} (k):= w_j(k,1) b_{i1} + \dots + w_j(k,N) b_{iN}$
		\EndFor
	\EndFor
\EndFor
\For {k=1 < N }
	\For {i= 1 < N }
		\For {j= 1 < N }
			\State $\mathcal{Z}(i,j,k) = \mathcal{W}_{j,k}(1) c_{i1}  + \dots +  \mathcal{W}_{j,k}(N) c_{iN}$ 
		\EndFor
	\EndFor
\EndFor
\end{algorithmic}
\end{mdframed}
\end{frame}

\subsection{HOSVD}

\begin{frame}
\frametitle{Kerntensor invertieren}
\begin{framed}
\begin{equation*}
\pmb{\mathcal{X}}^{+} = \pmb{\mathcal{G}}^{+} \times_{n=1}^{4} U^{ (n) ^{T} }.
\end{equation*}
\end{framed}

Wie schwer ist es die Pseudoinversen des Kerntensor $\mathcal{G}$ zu finden?
Kerntensor ist in der Regel vollbesetzt. \\ \textbf{Aber} der Kerntensor enthält zwei Arten von Zahlen. Sehr kleine Zahlen kleiner als $10^{-10}$ und Zahlen größer $1$. \\
\textbf{Idee} \\
Setzte alle kleinen Zahlen auf 0 $\rightarrow$ Der Kerntensor ist nun diagonal und sehr einfach zu invertieren.

\end{frame}

\begin{frame}
\frametitle{Tensor-Vektor Produkt}
\begin{framed}
\textbf{Definition} Tensor-Vektor Produkt \\
Es sei $\mathcal{A} \in \mathbb{R}^{N \times N \times N \times N}$ und $U \in \mathbb{R}^{N \times N}$.
Dann ist $Y=tvp(\mathcal{A},U)$ gegeben durch


\begin{equation*}
Y_{i_1 \, i_2} = \sum\limits_{j_1=1}^{N} \sum\limits_{j_2=1}^{N} \mathcal{A}_{i_1 i_2 j_1 j_2} U_{j_1 j_2}
\end{equation*}

\end{framed}

$\rightarrow$ \textbf{Komplexität} $O(N^4)$.

\end{frame}

\begin{frame}
\frametitle{Pseudoinversen-Tensor Vektor Produkt}
\textbf{1.Möglichkeit}
\begin{equation*} \label{eq:pinv}
tvp(\mathcal{X}^{+},U) = \underbrace{tvp(\mathcal{G}^{+} \times_{n=1}^{4} U^{ (n) ^{T} },U)}_{O(3N^4+17N^5)}
\end{equation*} 


\textbf{2.Möglichkeit}
\begin{equation*} \label{eq:pinvcase}
\begin{aligned}
\mathcal{X}^{+}_{(n)} u&= \underbrace{U^{ (n) }  \mathcal{G}^{+}_{(n)} ( U^{ (N_{1})  } \otimes U^{ (N_{2}) }  \otimes U^{ (N_{3}) }) u}_{O(9N^4+11N^5)} \,.
\end{aligned}
\end{equation*}

\end{frame}

\subsection{Tensorstruktur}

\begin{frame}
\frametitle{Massenmatrix}
\begin{framed}
\begin{equation*}
M^+ u =(\mathcal{W}_N^{1D} (\mathcal{N}^{1D})^T)^+ \otimes  (\mathcal{W}_N^{1D} (\mathcal{N}^{1D})^T)^+  u
\end{equation*}
\end{framed}
\begin{enumerate}
\item Berechnung des Matrix-Matrix Produkts $O(N^3)$
\item Berechnung der Pseudoinversen/Inversen mittels SVD oder Gauß Algorithmus von $\mathcal{W}_N^{1D} (\mathcal{N}^{1D})^T \in \mathbb{R}^{N \times N}$ ist von Komplexität $O(N^3)$
\item Matrix-Vektor Produkt dank unseres Algorithmus Komplexität von $O(4N^3)$
\end{enumerate}
\textbf{Insgesamt} $O(6N^3)$.
\end{frame}

\begin{frame}
\frametitle{Laplace Bilinearform}
\begin{framed}
\center{\textbf{Lyapunow Gleichung}} \\
\begin{equation*}
Y=AU+UA^T.
\end{equation*}
\end{framed}
Lösung ist gegeben durch

\begin{framed}
\begin{equation*}
U = \int\limits_{0}^{\infty} e^{A \tau} (-V) e^{A^T \tau} d\tau.
\end{equation*}
\end{framed}
\end{frame}

\begin{frame}
\frametitle{Lyapunow Gleichung}
\begin{itemize}
\item Naive Berechnung hat Komplexität von $O(N^6)$
\item Bartels-Stewart Algorithmus \cite{Bartels} liefert uns eine Komplexität von $O(N^3)$.
\end{itemize}
\end{frame}