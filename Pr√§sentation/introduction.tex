\begin{frame}
\frametitle{Hochleistungsrechnen}
\begin{framed}
\center{\textbf{Ziel} Löse ein sehr komplexes Problem.}
\end{framed} 
\center{ \textbf{Lösungsansatz} }
Teile das komplexe Problem auf in Subprobleme (Parallelisierung).
\end{frame}


\begin{frame}
\frametitle{Initial-Problem}
\begin{framed}
\begin{equation*}
v = A(u)
\end{equation*}
A, möglicherweise nichtlinearer, finite Elemente Operator, der Vektor u als Input nimmt.
\end{framed}
Probleme 
\begin{itemize}
\item $A$ wird unter Umständen sehr groß $\rightarrow$ Speicherplatz.
\item $A$ liegt nicht mehr im Cache $\rightarrow$ Abrufen der Elemente von $A$ zeitintesiv. 
\item Berechnung des Matrix-Vektor-Produkts komplex
\end{itemize}
\end{frame}

\begin{frame}
\frametitle{Divide and Conquer}
Nach \cite{Kronbichler} können wir die Ursprungsgleichung umformen zu
\begin{equation*}
v = A(u) = \sum\limits_{k=1}^{n_{cells}} P_k^T A_k P_k u \, .
\end{equation*}
$P_k$ kümmert sich um die Einordnug der lokalen Freiheitsgrade in die globalen Freiheitsgrade.
\begin{framed}
\begin{align*}
v_k &= A_k u_k \\
A_k^{-1} v_k &= u_k
\end{align*}
\end{framed}
\end{frame}

\begin{frame}
\frametitle{Inverse/Pseudoinverse}
\begin{enumerate}
\item Tensorstruktr und Summenfaktorisierung.
\item Singulärwertzerlegung höherer Ordnung (HOSVD).
\end{enumerate}

\end{frame}