\begin{Definition} Rang Eins Tensor \\
Ein Tensor ${\cal X}  \in \mathbb{R}^{I_1 \times I_2 \times \dots \times I_N}$ ist von Rang Eins wenn es als äußeres Produkt von N Vektoren
\begin{equation*}
{\cal{X}} = a^{(1)} \circ \dots a^{(N)}
\end{equation*}
geschrieben werden kann.
\end{Definition}

\begin{Bemerkung} Symmetrie \\
\begin{itemize}
\item Ein Tensor nennt man kubisch genau dann wenn jeder Mode dieselbe Dimension hat. 
\item Einen kubischen Tensor nennt man supersymmetrisch genau dann wenn die Elemente des Tensors konstant bleiben unter jeglicher Permutation der Indizes
\item Ein Tensor kann stückweise symmetrisch sein wenn die Elemente konstant bleiben unter der Permutation von mindestens 2 Indizes.
\end{itemize}
\end{Bemerkung}

\begin{Definition} Diagonal \\
Einen Tensor ${\cal X}  \in \mathbb{R}^{I_1 \times I_2 \times \dots \times I_N}$ nennt man diagonal, wenn
$x_{i_1,\dots,i_N} \neq 0$ genau dann wenn $i_1 = \dots = i_N$.
\end{Definition}

\begin{Bemerkung} Entfaltung \\
Einen Tenspr kann man entfalten. Dies impliziert eine Neu Ordnung der Tensorelemnte in eine Matrix.
Wir betrachten nur die sogenannte mode-n Entfaltung, da dies die einzig relevante Form der Entfaltung ist.
Eine mode-n Entfaltung eines Tensors ${\cal X}  \in \mathbb{R}^{I_1 \times I_2 \times \dots \times I_N}$ wird mit $\bold{X}_{(n)}$ geschrieben und ordnet die mode-n fibers in die Spalten der Ergebnismatrix.

\end{Bemerkung}