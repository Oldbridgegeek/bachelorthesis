(Numerik 2 Skript Kanschat)
Unser Ausgangspunkt ist nun
\begin{equation}
\text{Finde } u \in \Omega \text{ : } a(u,v) = f(v) \forall u \in \Omega \text{ und } v \in V
\end{equation}
Da die Sobolev Räume unendlich dimensional sind, ist es schwer sich mit der Konstruktion einer Lösung vertraut. Daher ist die Kernidee des Galerkin Verfahrens unseren Banachraum zu diskretisieren. 
Dazu führen wir eine sogenannte konforme Approximation durch.
Wir wählen $V_{n} \subset V$, sodass $dim V_{n} = n < \infty $. Nun gilt für die Minimierung in (\ref{eq:mini})
\begin{equation*}
\begin{aligned}
u &= \argmin_{v \in V} J(v)  \text{ (stetig)}\\
u_n &= \argmin_{v_n \in V_n} J(v_n) \text{ (diskret)}
\end{aligned}
\end{equation*}
Daraus folgt
\begin{equation*}
J(u_n) \geq J(u)
\end{equation*}
Dies folgt direkt aus der Wahl des Raumes als Teilraum des ursprünglichen Raumes. Man nennt diese Methode konforme Ritz-Galerkin Methode aus dem Grund, dass der diskrete Raum ein Teilraum von dem ursprüngichen Raum is und die Funktion J gleich bleibt. 
Was hat uns das Ganze gebracht? Nun da, $dim V_n = n < \infty$ können wir eine Basis für $V_n$. Das bringt uns den Vorteil, dass wir $u_n$ als Linear Kombination der Basiselemente in $V_n$ approximieren können.

Die schwache Formulierung sieht nun wie folgt aus
\begin{Lemma} \label{eq:weakgalerkin} Schwache Formulierung (Galerkin) \\
Finde $u_n \in V_n$, sodass $a(u_n,v_n)=f(v_n)$ $\forall v_n \in V_n$.
\end{Lemma}

Sei nun $e_1, \dots , e_n$ eine Basis von $V_n$. Es ist nun ausreichend nur die Basis zum Testen zu nutzen.
Die obere Gleichung in Lemma \ref{eq:weakgalerkin} reduziert sich auf
\begin{equation} \label{eq:weak1}
a(u_n,e_i) = f(e_i) \forall i \in \{1,\dots,n\}
\end{equation}

Im nächsten Schritt erweitern wir $u_n$ als Linearkombination wie folgt
\begin{equation}
u_n = \sum_{j=1}^{n} u_j e_j
\end{equation}

und setzen dies in \ref{eq:weak1} ein und erhalten.
\begin{equation}
\begin{aligned}
a( \sum_{j=1}^{n} u_j e_j, e_i ) &= f(e_i) \Longleftrightarrow \\
\sum_{j=1}^{n} u_j a( e_j, e_i ) &= f(e_i)
\end{aligned}
\end{equation}
Das können wir zusammenfassen in einem Linearen Gleichungssystem mit $A_{ij}=a(e_j,e_i)$ und $u=(u_1,\dots,u_n)^{T}$. Insgesamt erhalten wir
\begin{equation}
Au=f
\end{equation}

\begin{Bemerkung} Eigenschaften Galerkin \\
\begin{enumerate} 
\item Galerking Orthogonalität \\
Eine Kerneigenschaft der Galerkin Methode ist, dass der Fehler orthogonal zu dem Teilraum von V liegt.
\item Symmetrie \\
Die Matrix A ist genau dann symmetrisch, wenn die Bilinearform symmetrisch ist.
\end{enumerate}
\end{Bemerkung}

