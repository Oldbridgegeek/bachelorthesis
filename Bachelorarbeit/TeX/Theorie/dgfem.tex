Die diskontinuierliche Galerkin-Methode wurde 1973 erstmals von Reed und Hill eingeführt für hyperbolische Gleichung erster Ordnung. Es gab eine Reihe von Untersuchungen seither bezüglich hyperbolische Probleme erster Ordnung als auch für die Diskretisierung instationärer Probleme.
Unabhängig davon wurde die diskontinuierliche Galerkin-Methode für elliptische Gleichungen vorgeschlagen. Man wollte mehr Flexibilität bezüglich der Stetigkeitsvoraussetzungen an unsere lokalen Funktionen und mehr Freiraum bei der Gitter Generierung, insbesondere bei adaptiven h-p-Methoden. 
Die Art von dGFEM-Code erleichtert Parallelisierbarkeit was in Zeiten von GPU Programmierung eine große Effizienzsteigerung erlaubt. 
Die Idee von dGFEM ist maßgeblich Strafterme einzuführen, welche Unstetigkeit zwar erlauben aber in ihrem Ausmaß einschränkt. Diese Freiheit erlaubt uns aber zum Beispiel lokal Polynome höherer Ordnung zu benutzen und Singularitäten damit gekonnt zu beseitigen, ohne von Vorne zu beginnen zu müssen. Zu mal pro Element ein hängender Knoten erlaubt ist. D.h. eine Verfeinerung des Gitters ist lokal erlaubt, ohne direkt Probleme zu bekommen. (S.292 Grossmann)
