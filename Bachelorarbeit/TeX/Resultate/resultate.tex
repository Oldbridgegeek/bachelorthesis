Wir haben uns für die Berechnung der Pseudoinverse und das Produkt der Pseudoinversen mit einem beliebigen Vektor zwei Alternativen angeschaut. Die eine nutzt die Tensorstruktur der gegebenen Bilinearform, während die andere die Matrizen der Bilinearform in Tensoren umdefiniert und dann eine Singulärwertzerlegung höherer Ordnung durchführt.

Es ist naheliegend, dass man den Ansatz mit der Singulärwertzerlegung wählt, da dieser universell ist. Wir brauchen kein Vorwissen über unsere Bilinearform, insbesondere keine strukturspezifischen Informationen.
Doch die Arbeit hat gezeigt, dass dieser Ansatz mit einer so hohen Komplexität verbunden ist, dass er sich nicht rentiert.

Der erste Ansatz versucht die Elementsteifigkeitsmatrizen erst in eine Tensorstruktur zu überführen. Dieser Ansatz ist nicht universell und muss für jede Bilinearform extra hergeleitet werden. Mit Hilfe eines Algorithmuses für die effiziente Berechnung eines Matrix-Vektor Produkts mit einer Matrix die eine Tensorstruktur hat, ist es möglich eine sehr niedrige Komplexität zu erzielen. Wir sprechen hier von einer kubischen Ordnung abhängig vom Polynomgrad. Im Vergleich dazu hat der zweite Ansatz die Komplexitätsklasse $O(N^{10})$, wobei $N$ der Polynomgrad ist. 


Zu dem braucht der Ansatz der HOSVD die Definition und Speicherung der Elementsteifigkeitsmatrix der korrespondierenden Bilinearform, was wir versuchen sollten zu vermeiden. Der Grund liegt nicht in der Ersparnis von Speicherplatz. Sondern darin, dass das Abrufen von Elementen dieser Elementsteifigkeitsmatrix, wenn diese so groß ist, dass sie nicht mehr im Cache abgelegt werden kann, mit einem überproportionalen Zeitaufwand verbunden ist. Dies sollte gemieden werden.

Man kann auf die Ergebnisse dieser Arbeit aufbauend versuchen im ersten Ansatz die Komplexität noch weiter zu drücken. Dies kann man über eine \textit{trunkierte} oder andere Formen der Singulärwertzerlegung probieren. Man kann eine effiziente Matrix-Matrix Multiplikation, wie zum Beispiel mit dem \textit{Strassen-Algorithmus} nutzen. Für die Laplace Bilinearform kann man sich effizientere Algorithmen für die Berechnung der Lyapunow Gleichung anschauen.


